\documentclass{article}

\usepackage{commath}

\begin{document}
  current-current two-point functions
  \begin{equation}
    \begin{split}
      \Pi_{\mu\nu}(q) &= i \int \dif^4 x e^{iqx} \langle  0 | T \left\{ \mathcal{J}_{ij}^{\mu}(x) \mathcal{J}_{ij}^{\nu\dagger}(0) \right\} \rangle \\
      &= \left(q_\mu q_\nu = q^2 g_{\mu\nu}\right) \Pi_{ij,\mathcal{J}}^{(1)}(q^2) + q^\mu q^\nu \Pi_{ij,\mathcal{J}}^{(0)} (q^2) \\
      &= \left( q_\mu q_\nu - q^2 g_{\mu\nu} \right) \Pi^{(1+0)}(q^2) + q^2 g_{\mu\nu} \Pi^{(0)}(q^2)
    \end{split}
  \end{equation}
  
  Inclusive ratio:
  \begin{equation}
    R_\tau = \frac{\Gamma [ \tau^- \to \nu_\tau \text{hadrons}]}{\Gamma [ \tau^- \to \nu_\tau e^- \bar\nu_e]}
  \end{equation}

  \begin{equation}
    R_{\tau} = 12 \pi S_{EW} \int_0^{m_\tau} \frac{\dif s}{m_\tau^2} \left( 1 - \frac{s}{m_\tau^2}\right) \left[ \left( 1+2\frac{s}{m_\tau^2} \right) \operatorname{Im} \Pi^{(1)}(s) + \operatorname{Im} \Pi^{(0)}(s)\right]
  \end{equation}

  \begin{equation}
    \Pi^{(J)}(s) \equiv \envert{V_{uq}}^2 \left( \Pi_{ud,V}^{(J)} + \Pi_{ud,A}^{(J)}(s) \right)
  \end{equation}
  
  OPE:
  \begin{equation}
    \Pi_{OPE}^{(1+0)}(s) = \sum_{k=0}^{\infty}\frac{C_{2k}(s)}{(-s)^k}
  \end{equation}
  

  Countour integral:
  \begin{equation}
    \int_{s_{th}}^{s_0} \frac{\dif s}{s_0} \omega(s) \operatorname{Im} \Pi_{V/A}(s) = \frac{i}{2} \oint_{\envert{s}=s_0} \frac{\dif s}{s_0} \omega(s) \Pi_{V/A}(s)
  \end{equation}

  chisquared:
  \begin{equation}
    \chi^2(\alpha) = (I_i^{exp}-I_i^{th}(\alpha)) C_{ij}^{-1} (I_j^{exp}-I_j^{th}(\alpha))
  \end{equation}

  used weights
  \begin{equation}
    \omega_{wk}(s) = \left( 1 - \frac{s}{m_\tau^2} \right)^{2+k} \left( \frac{s}{m_\tau^2} \right)^l \left( 1 + \frac{2s}{m_\tau^2} \right)
  \end{equation}
  $(k,l) = {(0,0), (1,0), (1,1), (1,2), (1,3)}$

  weight to OPE:
  \begin{equation}
    \frac{1}{2 \pi i s_0} \oint_{\envert{s}=s_0} \dif s \left( \frac{s}{s_0} \right)^n \frac{C_{2k}}{(-s)^k} = (-1)^{n+1} \frac{C_{2(n+1)}}{s_0^{n+1}} \delta_{k,n+1}
  \end{equation}
  implying that an $n$-th degree monomial in the weight $\omega(s/s_0)$ selects
  the $D=2k=2(n+1)$ term in the OPE.

  \begin{equation}
    \begin{split}
      A_{00,V/A}^{ALEPH} &= A_{00,V/A}^{ALEPH}(a_s, \mathcal{O}_{6,V/A}) \\
      A_{10,V/A}^{ALEPH} &= A_{10,V/A}^{ALEPH}(a_s,\langle a_s GG \rangle, \mathcal{O}_{6,V/A}, \mathcal{O}_{10,V/A}) \\
      A_{11,V/A}^{ALEPH} &= A_{11,V/A}^{ALEPH}(a_s,\langle a_s GG \rangle, \mathcal{O}_{6,V/A}, \mathcal{O}_{10,V/A} \mathcal{O}_{12,V/A}) \\
      A_{12,V/A}^{ALEPH} &= A_{12,V/A}^{ALEPH}(a_s, \mathcal{O}_{6,V/A}, \mathcal{O}_{10,V/A}, \mathcal{O}_{12,V/A}, \mathcal{O}_{14,V/A}) \\
      A_{13,V/A}^{ALEPH} &= A_{13,V/A}^{ALEPH}(a_s, \mathcal{O}_{10,V/A}, \mathcal{O}_{12,V/A}, \mathcal{O}_{14,V/A}, \mathcal{O}_{16,V/A})
    \end{split} 
  \end{equation}

  $\alpha_s$
  \begin{center}
    \begin{tabular}{| c | c | c |}
      \hline
      Channel & $\alpha_s(m_\tau^2)$ & $\langle a_sGG \rangle$ \\ 
      \hline
      V+A (FOPT) & $0.319^{+0.010}_{-0.009}$ & -3 \\  
      \hline
      V+A (CIPT) & 0.339 & -16 \\    
      \hline
    \end{tabular}
  \end{center}
  
\end{document}