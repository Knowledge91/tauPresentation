\documentclass{article}

\usepackage{commath}
\usepackage{graphics} % scale tabular

\begin{document}
  current-current two-point functions
  \begin{equation}
    \begin{split}
      \Pi_{\mu\nu}(q) &= i \int \dif^4 x e^{iqx} \langle  0 | T \left\{ \mathcal{J}_{ij}^{\mu}(x) \mathcal{J}_{ij}^{\nu\dagger}(0) \right\} \rangle \\
      &= \left(q_\mu q_\nu - q^2 g_{\mu\nu}\right) \Pi_{ij,\mathcal{J}}^{(1)}(q^2) + q^\mu q^\nu \Pi_{ij,\mathcal{J}}^{(0)} (q^2) 
    \end{split}
  \end{equation}
  
  Inclusive ratio:
  \begin{equation}
    R_\tau = \frac{\Gamma [ \tau^- \to \nu_\tau \text{hadrons}]}{\Gamma [ \tau^- \to \nu_\tau e^- \bar\nu_e]}
  \end{equation}

  \begin{equation}
    R_{\tau} = 12 \pi S_{EW} \int_0^{m_\tau} \frac{\dif s}{m_\tau^2} \left( 1 - \frac{s}{m_\tau^2}\right) \left[ \left( 1+2\frac{s}{m_\tau^2} \right) \operatorname{Im} \Pi^{(1)}(s) + \operatorname{Im} \Pi^{(0)}(s)\right]
  \end{equation}

  \begin{equation}
    \Pi^{(J)}(s) \equiv \envert{V_{uq}}^2 \left( \Pi_{ud,V}^{(J)} + \Pi_{ud,A}^{(J)}(s) \right)
  \end{equation}
  
  OPE:
  \begin{equation}
    \Pi_{OPE}^{(1+0)}(s) = \sum_{k=0}^{\infty}\frac{C_{2k}(s)}{(-s)^k}
  \end{equation}

  \begin{equation}
    \mathcal{O}_{4,V/A} = \frac{1}{12}\left[ 1 - \frac{11}{18} a_s \right] \langle a_s GG \rangle + \left[ 1 + \frac{\pm 36 - 23}{27} a_s \right] \langle (m_u + m_d) \bar q q \rangle
  \end{equation}

  D=6:
  Anomalous Dimension Matrix (V-A):
  \begin{equation}
    \hat \gamma^{(1)}_{Q_-} = 
    \begin{pmatrix}
      -\frac{3N_C}{2}+\frac{3}{N_C} & -\frac{3C_F}{2N_C} \\
      -3 & 0
    \end{pmatrix}
  \end{equation}

  Anomalous Dimension Matrix (V+A):
  \begin{equation}
    \begin{split}
      \label{eq:anomalousDimensionMatrixVpA}
      \gamma^{(1)}_{Q_+} &= 
      \left(\begin{matrix}
          -\frac{3}{N_C} & \frac{3C_F}{2N_C} &-\frac{1}{3N_C} & 0   
          \\
          3 & 0 & \frac{2}{3} & 0    \\
          0 & 0 & \frac{N_f}{3}-\frac{3N_c}{4}-\frac{1}{3N_c} & \frac{3N_c}{4}-\frac{3}{N_c}   \\
          \frac{3}{2}+\frac{3}{2N_c} & -\frac{3C_F}{2N_c} & \frac{3N_c}{4}+\frac{3}{2}-\frac{11}{6N_c} & -\frac{3N_c}{4}+\frac{3}{2}+\frac{3}{2N_c} \\
          0 & 0 & \frac{11}{3} & 0 \\
          0 & 0 & 0 & 0 \\
          0 & 0 & 0 & 0 \\
          0 & 0 & 0 & 0 \\
          0 & 0 & 0 & 0 
        \end{matrix} \right. \\
      &\qquad\qquad\qquad\qquad\qquad\qquad \left.\begin{matrix}
          0 & 0 & 0 & 0 & 0 \\
          0 & 0 & 0 & 0 & 0 \\
          \frac{3C_F}{2N_C} & \frac{2}{3} & 0 & 0 & 0 \\
          -\frac{3C_F}{2N_c} & -\frac{3}{4}-\frac{3}{4N_c} & -\frac{3}{4}-\frac{3}{4N_c} & \frac{3C_F}{4N_c} & \frac{3C_F}{4N_c} \\
          0 & 0 & 0 & 0 & 0 \\
          0 & \frac{N_f}{3}+1-\frac{3N_c}{4}-\frac{1}{3N_c} & \frac{3N_c}{4}-\frac{3}{N_c} & 0  & \frac{3C_F}{2N_c} \\
          0 & \frac{3N_c}{4}-\frac{10}{3N_c} & -\frac{3N_c}{4} & \frac{3C_F}{2N_c} & 0 \\
          0 & \frac{2}{3} & 3 & 0 & 0 \\
          0 & \frac{11}{3} & 0 & 0 & 0 
        \end{matrix}\right)
    \end{split}
  \end{equation}

  non-diagonal V/A current
  \begin{equation}
    j_\mu^V(x) = (\bar u \gamma_\mu d)(x)
  \end{equation}
  \begin{equation}
    j_\mu^A(x) = (\bar u \gamma_\mu \gamma_5 d)(x)
  \end{equation}

  Countour integral:
  \begin{equation}
    \int_{s_{th}}^{s_0} \frac{\dif s}{s_0} \omega(s) \operatorname{Im} \Pi_{V/A}(s) = \frac{i}{2} \oint_{\envert{s}=s_0} \frac{\dif s}{s_0} \omega(s) \Pi_{V/A}(s)
  \end{equation}

  adler function
  \begin{equation}
    D(s) \equiv -s \frac{\dif \Pi^{PT}}{\dif s} = \frac{1}{12\pi^2}\sum_{n=0}^\infty a_\mu^n \sum_{k=1}^{n+1} k c_{nk} \ln\left( \frac{-s}{\mu^2} \right)
  \end{equation}

  contour integral in terms of adler
  \begin{equation}
    A^{\omega,PT} = \frac{i}{2 s_0} \oint_{\envert{s}=s_o} \frac{\dif s}{s} \left[ W(s) - W(s_0) \right] D(s)
  \end{equation}

  spectral function:
  \begin{equation}
    \rho(s) \equiv \frac{1}{\pi} \operatorname{Im}\Pi(s)
  \end{equation}

  normalized invariant mass-squared distribution
  \begin{equation}
    \left( \frac{1}{N_{V/A}} \right)\left( \frac{\dif N_{V/A}}{\dif s} \right)
  \end{equation}
  
  aleph:
  \begin{equation}
    \begin{split}
      v_1(s) &\equiv \frac{m_\tau^2}{6 \envert{V_{ud}}^2S_{EW}} \frac{B(\tau^- \to V^- \nu_\tau)}{B(\tau^- \to e^- \bar \nu_e \nu_\tau)} \frac{\dif N_V}{N_V \dif s} \left[ \left( 1 - \frac{s}{m_\tau^2} \right)^2 \left( 1 + \frac{2 s}{m_\tau^2}\right) \right]^{-1} \\
      a_1(s) &\equiv \frac{m_\tau^2}{6 \envert{V_{ud}}^2S_{EW}} \frac{B(\tau^- \to A^- \nu_\tau)}{B(\tau^- \to e^- \bar \nu_e \nu_\tau)} \frac{\dif N_A}{N_A \dif s} \left[ \left( 1 - \frac{s}{m_\tau^2} \right)^2 \left( 1 + \frac{2 s}{m_\tau^2}\right) \right]^{-1} \\
      a_0(s) &\equiv \frac{m_\tau^2}{6 \envert{V_{ud}}^2S_{EW}} \frac{B(\tau^- \to \pi^- \nu_\tau)}{B(\tau^- \to e^- \bar \nu_e \nu_\tau)} \frac{\dif N_A}{N_A \dif s} \left( 1 - \frac{s}{m_\tau^2} \right)^2
    \end{split}
  \end{equation} 

  \begin{equation}
    \begin{split}
      \operatorname{Im} \Pi_{\bar u d,V}^{(1)}(s) &= \frac{1}{2\pi} v_1(s) \\
      \operatorname{Im} \Pi_{\bar u d,A}^{(1)}(s) &= \frac{1}{2\pi} a_1(s) \\
      \operatorname{Im} \Pi_{\bar u d,A}^{(0)}(s) &= \frac{1}{2\pi} a_0(s)
    \end{split}
  \end{equation}

  chisquared:
  \begin{equation}
    \chi^2(\alpha) = (I_i^{exp}-I_i^{th}(\alpha)) C_{ij}^{-1} (I_j^{exp}-I_j^{th}(\alpha))
  \end{equation}
  \begin{equation}
    \begin{split}
      I_{i=kl}^{exp}(s_k,\omega_l) &= \int_{s_{th}}^{s_k} \frac{\dif s}{s_k} \omega_l (s) \operatorname{Im} \Pi_{V/A}(s) \\
      I_{i=kl}^{th}(s_k,\omega_l) &= \frac{i}{2 s_k} \oint_{\envert{s}=s_k} \frac{\dif s}{s} \left[ W_l(s) - W_l(s_k) \right] D(s)
    \end{split}
  \end{equation} 
  
  used weights
  \begin{equation}
    \omega_{kl}(s) = \left( 1 - \frac{s}{m_\tau^2} \right)^{2+k} \left( \frac{s}{m_\tau^2} \right)^l \left( 1 + \frac{2s}{m_\tau^2} \right)
  \end{equation}
  $(k,l) = {(0,0), (1,0), (1,1), (1,2), (1,3)}$

  weight to OPE:
  \begin{equation}
    \frac{1}{2 \pi i s_0} \oint_{\envert{s}=s_0} \dif s \left( \frac{s}{s_0} \right)^n \frac{C_{2k}}{(-s)^k} = (-1)^{n+1} \frac{C_{2(n+1)}}{s_0^{n+1}} \delta_{k,n+1}
  \end{equation}
  implying that an $n$-th degree monomial in the weight $\omega(s/s_0)$ selects
  the $D=2k=2(n+1)$ term in the OPE.
  \begin{equation}
    \frac{s}{s_0} \quad + \quad \left( \frac{s}{s_0} \right)^2 \quad + \quad \left( \frac{s}{s_0} \right)^3 \quad + \quad \cdots
  \end{equation}
  
  \begin{equation}
    \begin{split}
      A_{00,V/A}^{ALEPH} &= A_{00,V/A}^{ALEPH}(a_s, \mathcal{O}_{6,V/A}) \\
      A_{10,V/A}^{ALEPH} &= A_{10,V/A}^{ALEPH}(a_s,\langle a_s GG \rangle, \mathcal{O}_{6,V/A}, \mathcal{O}_{10,V/A}) \\
      A_{11,V/A}^{ALEPH} &= A_{11,V/A}^{ALEPH}(a_s,\langle a_s GG \rangle, \mathcal{O}_{6,V/A}, \mathcal{O}_{10,V/A} \mathcal{O}_{12,V/A}) \\
      A_{12,V/A}^{ALEPH} &= A_{12,V/A}^{ALEPH}(a_s, \mathcal{O}_{6,V/A}, \mathcal{O}_{10,V/A}, \mathcal{O}_{12,V/A}, \mathcal{O}_{14,V/A}) \\
      A_{13,V/A}^{ALEPH} &= A_{13,V/A}^{ALEPH}(a_s, \mathcal{O}_{10,V/A}, \mathcal{O}_{12,V/A}, \mathcal{O}_{14,V/A}, \mathcal{O}_{16,V/A})
    \end{split} 
  \end{equation}


  duality violations
  \begin{equation}
    \Delta \rho_{V/A}^{DV}(s) = e^{-\delta_{V/A}+\gamma_{V/A}s} \sin(\alpha_{V/A} + \beta_{V/A}s)
  \end{equation}
  \begin{equation}
    \Delta A_{V/A}^{\omega,DV}(s_0) \equiv \frac{i}{2} \oint_{\envert{s_0}=s_0} \frac{\dif s}{s_0} \omega(s) \left\{ \Pi_{V/A}(s) - \Pi_{V/A}^{OPE}(s) \right\} = -\pi \int_{s_0}^\infty \frac{\dif s}{s_0} \omega(s) \Delta \rho_{V/A}^{DV}(s)
  \end{equation}

  \begin{center}

    Pich 2016
    $\alpha_s$ V+A from five weights
    {\renewcommand{\arraystretch}{1.5}
      \begin{tabular}{| c | c | c | c | c |}
        \hline
        Channel & $\alpha_s(m_\tau^2)$ & $\langle a_sGG \rangle$
        & $\mathcal{O}_6$ & $\mathcal{O}_8$ \\ 
                & & $(10^{-3} GeV^4)$ & $(10^{-3} GeV^6)$ & $(10^{-3} GeV^8)$\\
        \hline
        V+A (FOPT) & $0.319^{+0.010}_{-0.006}$ & $-3^{+6}_{-11}$
        & $1.3^{+1.4}_{-0.8}$ & $-0.8^{+0.4}_{-0.7}$ \\  
        \hline
        V+A (CIPT) & $0.339^{+0.011}_{-0.009}$ & $-16^{+5}_{-5}$
        & $0.9^{+0.3}_{-0.4}$ & $-1.0^{0.5}_{-0.7}$\\    
        \hline
      \end{tabular}
    }
    

    Boito 2015
    $\alpha_s$ V+A from three weights, multiple s0 starting by $s_{min}$
    \resizebox{\textwidth}{!}{
      {\renewcommand{\arraystretch}{1.5}
        \begin{tabular}{| c | c | c | c | c | c | c | c | c |}
          \hline
          Channel & $s_{min} (GeV^2)$ & $\alpha_s(m_\tau^2)$ & $\mathcal{O}_{6V,A}$ & $\mathcal{O}_{8V,A}$ & $\delta_{V,A}$ & $\gamma_{V,A}$ & $\alpha_{V,A}$ & $\beta_{V,A}$ \\ 
          \hline
          V+A (FOPT) & $1.550$ & $0.292(9)$ & $-0.90(13)$ & $1.57(22)$ & $3.19(51)$ & $0.80(30)$ & $-2.65(79)$ & $4.42(41)$ \\
                  & & & $-0.63(61)$ & $3.0(2.2)$ & $1.53(56)$ & $1.42(24)$ & $5.73(84)$ & $1.84(43)$ \\ 
          \hline
          V+A (CIPT) & $1.550$ & $0.312(13)$ & $-0.90(13)$ & $1.48(25)$ & $3.35(49)$ & $0.70(29)$ & $-2.28(81)$ & $4.23(42)$ \\
          &&& $1.59(55)$ & $1.44(25)$ & $5.37(89)$ & $2.03(46)$ & $-0.33(56)$ & $2.0(1.8)$ \\
          \hline
        \end{tabular}
      }
    }

  \end{center}
\end{document}